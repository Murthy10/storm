%!TEX root = ../dissertation.tex
\chapter{Installation}
\newpage

\section{Repositories}
For the implementation part I used different Github repositories to store the code.
And help the user with the deployment and installation of the applications.

\subsection{OSMstream}
The \fnurl{OSMstream}{https://github.com/geometalab/OSMstream} repository provides a docker container which collects the augmented diffs from OpenStreetMap.
For further use we convert the diffs from XML to JSON format and publish them with Apache Kafka. \\
\medskip
Requirements:
\begin{itemize}
    \item Scala 2.11
    \item Apache Kafka 0.10.0.1
    \item Python3
    \item Pip3
    \item or Docker \textgreater= 1.10
\end{itemize}
\newpage
\subsection{storm}
\fnurl{storm}{https://github.com/Murthy10/storm} is the main repository and hosts the Apache Storm implementation code with all the queries and a docker container for a simple usage.
This application uses the data provided by OSMstream.\\
\medskip
Requirements:
\begin{itemize}
    \item OSMstream
    \item Apache Storm 1.0.2
    \item JVM \textgreater= 1.8
    \item Maven \textgreater= 3.3.3
    \item or Docker \textgreater= 1.10
\end{itemize}


\subsection{StormBoard}
The \fnurl{storm}{https://github.com/Murthy10/StormBoard} is a simple live streaming web dasboard to visualise the queries.
It is based on a Node.js server, Socket.io for realtime data exchange and Angular 2.0 on the client side.\\
\medskip
Requirements:
\begin{itemize}
    \item Node.js >= 6.0.0
    \item AngularJS 2.0
    \item or Docker \textgreater= 1.10
\end{itemize}
